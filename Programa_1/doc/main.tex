% Materia: Fundamentos de Sistemas Embebidos
% Fecha de creación: 17/09/2021
% Descripción: Cuestionario del programa 1 sobre PWM
% Hecho por: Brito Segura Angel
                % Tamaño de hoja y letra (11pt o 12pt -10pt por default-)
\documentclass[letterpaper,11.5pt]{article} %Tipo de documento

% Paquetes en otro archivo
% %% %%%%%%%%%%%%%%%%%%%%%%%%%%%%%%%%%%%%%%%%%%%%%%%%%%%%%%%%%
%
% Materia: Fundamentos de Sistemas Embebidos
% Fecha de creación: 30/11/2021
% Descripción: Lista de paquetes requeridos
% Hecho por: Brito Segura Angel
%
% %% %%%%%%%%%%%%%%%%%%%%%%%%%%%%%%%%%%%%%%%%%%%%%%%%%%%%%%%%%
% Archivo principal de LaTeX
%!TEX root = ../main.tex
\usepackage[utf8]{inputenc} %Decodificación del documento
\usepackage[T1]{fontenc} %Agregar LATIN1 para todas las lenguas romances aceptadas
\usepackage[spanish,mexico]{babel} %Traducir a español mexicano
\usepackage{booktabs}
\usepackage{csquotes}
\usepackage{fancyhdr}
\usepackage{geometry}
\usepackage{graphicx}
\usepackage{lastpage}
\usepackage[all]{nowidow}
\usepackage[inline]{enumitem}
\usepackage[usenames,dvipsnames]{xcolor}
\usepackage{varioref}
\usepackage[hidelinks]{hyperref}
\usepackage[noabbrev,nameinlink,spanish]{cleveref}
\usepackage[square, comma, numbers, sort&compress]{natbib}


\title{Programa 1}
\author{\textbf{Brito Segura Angel}}
\date{Fecha de entrega: \today}

\begin{document}
    \maketitle %Poner el título, autor y fecha al documento
    
    \pagestyle{empty} %Quitar paginación
    \maketitle
    \thispagestyle{empty}
    
    \section{Cuestionario}
    \texttt{Investigue el efecto que se produciría al regular el tiempo de encendido de un LED\\mediante la modulación del ciclo de trabajo del PWM en alta frecuencia (ej. 1kHz) y\\explique por qué no es posible observar este efecto en los simuladores.}\\[1.5em]    
    La modulación de ancho de pulso (PWM) es una técnica en la cual se modifica el ciclo de trabajo de una señal periódica utilizando una fuente digital para crear una de tipo analógica. El ciclo de trabajo (\textit{dutycyle}) describe la cantidad de tiempo que la señal está en un estado alto (encendido) como un porcentaje del tiempo total que se tarda en completar un ciclo (\cite{tinkercard}).\\

    Al probar frecuencias altas en el ciclo de trabajo de PWM en el simulador RPiVirtualBoard no se observa dicho cambio y el comportamiento del circuito es igual a como se observa con frecuencias bajas: en 50\% el led parpadea (sin cambio de tiempos entre encendido y apagado) y en 100\% se mantiene encendido. Mientras que en el simulador en línea, entre mayor sea el ciclo de trabajo, parecería que el led se mantiene fijo (parpadea muy poco).\\
    
    Recordar que las bibliotecas de código abierto no siempre son fiables, justo en su licencia se indica esto, por lo que si bien los simuladores ocupando este tipo de librerías son una herramienta poderosa y necesaria en el diseño de circuitos inteligentes; se recomienda realizar pruebas de comprobación a los modelos empleados en el diseño antes de confiar plenamente en los resultados obtenidos (\cite{arduino}). Las condiciones en las que corren estas herramientas son valores teóricos, por lo que no es posible observar efectos con alta frecuencia al ocupar una parte aislada de nuestra computadora al ocuparlos. El hacer una implementación física nos permite ocupar realmente el procesador de la tarjeta propuesta en el diseño con lo que se logra alcanzar y visualizar el efecto de cambio en el LED con frecuencias de más de 1 kHz.\newline
    
    \printbibliography %Imprimir bibliografía
\end{document}
