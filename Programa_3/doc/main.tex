% %% %%%%%%%%%%%%%%%%%%%%%%%%%%%%%%%%%%%%%%%%%%%%%%%%%%%%%%%%%
% Materia: Fundamentos de Sistemas Embebidos
% Fecha de creación: 17/10/2021
% Descripción: Cuestionario del programa 3
% Hecho por: Brito Segura Angel
% %% %%%%%%%%%%%%%%%%%%%%%%%%%%%%%%%%%%%%%%%%%%%%%%%%%%%%%%%%%
% Raíz del proyecto (este archivo)
%!TEX root = ./main.tex
% Archivo de referencias bibliográficas
%!TEX root = ./referencias.bib

                % Tamaño de hoja y letra (11pt o 12pt -10pt por default-)
\documentclass[letterpaper,10.5pt]{article} %Tipo de documento

% Paquetes (importación de librerías)
% %% %%%%%%%%%%%%%%%%%%%%%%%%%%%%%%%%%%%%%%%%%%%%%%%%%%%%%%%%%
%
% Materia: Fundamentos de Sistemas Embebidos
% Fecha de creación: 30/11/2021
% Descripción: Lista de paquetes requeridos
% Hecho por: Brito Segura Angel
%
% %% %%%%%%%%%%%%%%%%%%%%%%%%%%%%%%%%%%%%%%%%%%%%%%%%%%%%%%%%%
% Archivo principal de LaTeX
%!TEX root = ../main.tex
\usepackage[utf8]{inputenc} %Decodificación del documento
\usepackage[T1]{fontenc} %Agregar LATIN1 para todas las lenguas romances aceptadas
\usepackage[spanish,mexico]{babel} %Traducir a español mexicano
\usepackage{booktabs}
\usepackage{csquotes}
\usepackage{fancyhdr}
\usepackage{geometry}
\usepackage{graphicx}
\usepackage{lastpage}
\usepackage[all]{nowidow}
\usepackage[inline]{enumitem}
\usepackage[usenames,dvipsnames]{xcolor}
\usepackage{varioref}
\usepackage[hidelinks]{hyperref}
\usepackage[noabbrev,nameinlink,spanish]{cleveref}
\usepackage[square, comma, numbers, sort&compress]{natbib}

% Macros (creación de nuevos comandos y entornos)
% %% %%%%%%%%%%%%%%%%%%%%%%%%%%%%%%%%%%%%%%%%%%%%%%%%%%%%%%%%%
%
% Materia: Fundamentos de Sistemas Embebidos
% Fecha de creación: 30/11/2021
% Descripción: Creación de nuevos comandos
% Hecho por: Brito Segura Angel
%
% %% %%%%%%%%%%%%%%%%%%%%%%%%%%%%%%%%%%%%%%%%%%%%%%%%%%%%%%%%%
% Archivo principal de LaTeX
%!TEX root = ../main.tex
\geometry {
	margin=2cm,
	top=3cm
}
\setlength{\parskip}{1em}
\linespread{1.3}
\pagestyle{fancy}
\fancyhf{} % remove everything
\renewcommand{\headrulewidth}{0pt} % remove lines as well
\rfoot{Página~\thepage~de~\pageref{LastPage}}
\rhead{
	Brito Segura Angel \\
	Fundamentos de Sistemas Embebidos --- Grupo 2
}


\title{Programa 3}
\author{\textbf{Brito Segura Angel}}
\date{Fecha de entrega: \today}

%Configuraciones del documento (cambios a nivel documento)
% %% %%%%%%%%%%%%%%%%%%%%%%%%%%%%%%%%%%%%%%%%%%%%%%%%%%%%%%%%%
%
% Materia: Fundamentos de Sistemas Embebidos
% Fecha de creación: DD/MM/2021
% Descripción: Configuración de página y documento en general
% Hecho por: Brito Segura Angel
%
% %% %%%%%%%%%%%%%%%%%%%%%%%%%%%%%%%%%%%%%%%%%%%%%%%%%%%%%%%%%
% Archivo principal de LaTeX
%!TEX root = ../main.tex

\makeatletter %Utilizar referencias con @
%Configuración de los hipervínculos y metadatos del PDF
\hypersetup {
	hidelinks,
	colorlinks=true,
	linkcolor=Black, %pone negro a la referencia de página, ver si no afecta en otros lados
	filecolor=OliveGreen, %\href{run:./file.txt}{File.txt}
	urlcolor=RoyalPurple,
	citecolor=Blue,
	pdfauthor={\@author},
	pdftitle={\@title},
%	pdfpagemode=FullScreen,
	pdfsubject={Fundamentos de Sistemas Embebidos},
	pdfkeywords={},
	pdfproducer={SublimeText3 con Makefile (Ubuntu)},
	pdfcreator={pdflatex, rubber}
}
\makeatother %Regresar a la configuración normal

%Cambiar el nombre de la refencia dada: tipo, en sigular, en plural
\crefname{section}{sección}{secciones}
\Crefname{section}{Sección}{Secciones}
%Agregar los que sean necesarios cambiar a los que trae por defecto
\crefname{equation}{ecuación}{ecuaciones}
\Crefname{equation}{Fórmula}{Fórmulas}


\begin{document}
    \pagestyle{fancy} %Estilo de página
    \maketitle %Poner el título, autor y fecha al documento
    
    %Para solo una página, descomentar estas líneas y quitar TODO lo de fancy
    %\pagestyle{empty} %Quitar paginación
    %\maketitle
    %\thispagestyle{empty}
    
    \section{Cuestionario}
    \subsection{¿Cuáles son los valores óptimos para las resistencias comerciales usadas en el divisor de voltaje que alimenta al pin $V_{Ref+}$? Justifique su respuesta e incluya el análisis matemático (cálculos) correspondiente}
	    Teniendo que un Arduino UNO es alimentado por 5 [V], su pin de $V_{Ref+}$ sería dicho valor y dejando fijo el valor de $R_2$, de acuerdo al divisor de voltaje, se tiene la siguiente fórmula:
	    \begin{equation}
			\label{eqn:ec1}
			V_{out} = \frac{R_2}{10~[k\Omega] + R_2} * 5~[V]
		\end{equation}
		El valor de $V_{out}$ depende directamente del rango de lecturas en grados centígrados que se desee obtener del sensor. Para este caso se desea que el rango máximo sea de 150°C, por lo que se puede resolver la \cref{eqn:ec1} para despejar el valor faltante de $R_1$:
		\begin{equation}
			\label{eqn:ec2}
			150[^{\circ} C]* 0.01\left[\frac{V}{^{\circ} C}\right] = \frac{R_2}{10~[k\Omega] + R_2} * 5~[V]
		\end{equation}
		Despejando $R_2$ de la \cref{eqn:ec2}:
		\[
			1.5 * (10000 + R_2) = 5R_2 \Rightarrow 15000 = 5R_2 - 1.5R_2 \Rightarrow 3.5R_2 = 15000 \Rightarrow R_2 = \frac{15000}{3.5}\approx 4286~[\Omega]
		\]

		Si bien el valor adecuado para $R_2$ es de $4.2~[k\Omega]$, en el mercado \cite{res_comerciales} no es posible tener una resistencia de este valor, por lo que \textbf{los valores óptimos para resistencias comerciales son: $R_1 = 10~[k\Omega]$ y $R_2 = 3.9~[k\Omega]$} para el divisor de voltaje propuesto para poder ocupar los 1024 valores que nos da este convertidor.\newline

		\subsection{Suponga que se tiene un ADC que puede opera a una frecuencia de hasta 1kHz. ¿Qué estrategia se recomienda para maximizar la presición del sensor? ¿Conviene seguir utilizando el promedio simple u otra técnica de filtrado? ¿Es conveniente reducir la frecuencia de muestreo? Justifique su respuesta}
		Lozano: \url{https://repositorio.unican.es/xmlui/bitstream/handle/10902/7633/Reducci%c3%b3n%20de%20la%20frecuencia.pdf?sequence=1&isAllowed=y}

		Montoro - Página 5: \url{https://idus.us.es/bitstream/handle/11441/69438/TFM_LOZANO%20FDEZ%20DIEGO.pdf}

		Serrato - Página 33: \url{https://hera.ugr.es/tesisugr/16072595.pdf}

	\hfill \break %Salto de línea	
	%Agregando fuentes de consulta
	\bibliographystyle{unsrtnat} %abbrnat es parecido a APA
	\bibliography{referencias}
    %\printbibliography %Imprimir bibliografía (si se ocupa bibtex y descomentar lo anterior)

\end{document}
