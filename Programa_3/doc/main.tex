% %% %%%%%%%%%%%%%%%%%%%%%%%%%%%%%%%%%%%%%%%%%%%%%%%%%%%%%%%%%
% Materia: Fundamentos de Sistemas Embebidos
% Fecha de creación: 17/10/2021
% Descripción: Cuestionario del programa 3
% Hecho por: Brito Segura Angel
% %% %%%%%%%%%%%%%%%%%%%%%%%%%%%%%%%%%%%%%%%%%%%%%%%%%%%%%%%%%
% Raíz del proyecto (este archivo)
%!TEX root = ./main.tex
% Archivo de referencias bibliográficas
%!TEX root = ./referencias.bib

                % Tamaño de hoja y letra (11pt o 12pt -10pt por default-)
\documentclass[letterpaper,10.5pt]{article} %Tipo de documento

% Paquetes (importación de librerías)
% %% %%%%%%%%%%%%%%%%%%%%%%%%%%%%%%%%%%%%%%%%%%%%%%%%%%%%%%%%%
%
% Materia: Fundamentos de Sistemas Embebidos
% Fecha de creación: 10/10/2021
% Descripción: Lista de paquetes requeridos
% Hecho por: Brito Segura Angel
%
% %% %%%%%%%%%%%%%%%%%%%%%%%%%%%%%%%%%%%%%%%%%%%%%%%%%%%%%%%%%
% Archivo principal de LaTeX
%!TEX root = ../main.tex

\usepackage[utf8]{inputenc} %Decodificación del documento
\usepackage[T1]{fontenc} %Agregar LATIN1 para todas las lenguas romances aceptadas
\usepackage[spanish,mexico]{babel} %Traducir a español mexicano
% Forma recomendada de citar
\usepackage[square, comma, numbers, sort&compress]{natbib} %De acuerdo a la IEEE

\usepackage{csquotes} %Diferentes tipos de citas

% Referencias en el documento
\usepackage{varioref} %ya no se ocupa pero es necesario para los siguientes
\usepackage{hyperref} %habilitar hipervínculos y modificarlos
\usepackage[noabbrev,nameinlink,spanish]{cleveref} %generar referencias

\usepackage[colalign]{aligncolsatbottom}  %Para la alineación de las columnas con cierto estilo

% Macros (creación de nuevos comandos y entornos)
% %% %%%%%%%%%%%%%%%%%%%%%%%%%%%%%%%%%%%%%%%%%%%%%%%%%%%%%%%%%
%
% Materia: Fundamentos de Sistemas Embebidos
% Fecha de creación: 30/11/2021
% Descripción: Creación de nuevos comandos
% Hecho por: Brito Segura Angel
%
% %% %%%%%%%%%%%%%%%%%%%%%%%%%%%%%%%%%%%%%%%%%%%%%%%%%%%%%%%%%
% Archivo principal de LaTeX
%!TEX root = ../main.tex
\geometry {
	margin=2cm,
	top=3cm
}
\setlength{\parskip}{1em}
\linespread{1.3}
\pagestyle{fancy}
\fancyhf{} % remove everything
\renewcommand{\headrulewidth}{0pt} % remove lines as well
\rfoot{Página~\thepage~de~\pageref{LastPage}}
\rhead{
	Brito Segura Angel \\
	Fundamentos de Sistemas Embebidos --- Grupo 2
}


\title{Programa 3}
\author{\textbf{Brito Segura Angel}}
\date{Fecha de entrega: \today}

%Configuraciones del documento (cambios a nivel documento)
% %% %%%%%%%%%%%%%%%%%%%%%%%%%%%%%%%%%%%%%%%%%%%%%%%%%%%%%%%%%
%
% Materia: Fundamentos de Sistemas Embebidos
% Fecha de creación: 30/11/2021
% Descripción: Configuración de página y documento en general
% Hecho por: Brito Segura Angel
%
% %% %%%%%%%%%%%%%%%%%%%%%%%%%%%%%%%%%%%%%%%%%%%%%%%%%%%%%%%%%
% Archivo principal de LaTeX
%!TEX root = ../main.tex

%Configuración de los hipervínculos y metadatos del PDF
\hypersetup {
	hidelinks,
	colorlinks=true,
	linkcolor=Black, %pone negro a la referencia de página, ver si no afecta en otros lados
	filecolor=OliveGreen, %\href{run:./file.txt}{File.txt}
	urlcolor=RoyalPurple,
	citecolor=Blue,
	pdfauthor={Brito Segura Angel},
	pdftitle={Reporte de lectura - Clean Code},
%	pdfpagemode=FullScreen,
	pdfsubject={Fundamentos de Sistemas Embebidos},
	pdfkeywords={},
	pdfproducer={SublimeText3 con Makefile (Ubuntu)},
	pdfcreator={pdflatex, rubber}
}


\begin{document}
    \pagestyle{fancy} %Estilo de página
    \maketitle %Poner el título, autor y fecha al documento
    
    \section{Cuestionario}
    \subsection{¿Cuáles son los valores óptimos para las resistencias comerciales usadas en el divisor de voltaje que alimenta al pin $V_{Ref+}$? Justifique su respuesta e incluya el análisis matemático (cálculos) correspondiente}
	    Teniendo que un Arduino UNO es alimentado por 5 [V], su pin de $V_{Ref+}$ sería dicho valor. Dejando fijo el valor de $R_2$, conforme al divisor de voltaje, se tiene la siguiente fórmula:
	    \begin{equation}
			\label{eqn:ec1}
			V_{out} = \frac{R_2}{10~[k\Omega] + R_2} * 5~[V]
		\end{equation}
		El valor de $V_{out}$ depende directamente del rango de lecturas en grados centígrados que se desee obtener del sensor. Para este caso, se desea que el rango máximo sea de 150°C, por lo que se puede resolver la \cref{eqn:ec1} para despejar el valor faltante de $R_1$:
		\begin{equation}
			\label{eqn:ec2}
			150[^{\circ} C]* 0.01\left[\frac{V}{^{\circ} C}\right] = \frac{R_2}{10~[k\Omega] + R_2} * 5~[V]
		\end{equation}
		Despejando $R_2$ de la \cref{eqn:ec2}:
		\[
			1.5 * (10000 + R_2) = 5R_2 \Rightarrow 15000 = 5R_2 - 1.5R_2 \Rightarrow 3.5R_2 = 15000 \Rightarrow R_2 = \frac{15000}{3.5}\approx 4286~[\Omega]
		\]

		Si bien el valor adecuado para $R_2$ es de $4.2~[k\Omega]$, en el mercado \cite{res_comerciales} no es posible tener una resistencia de este valor, por lo que \textbf{los valores óptimos para resistencias comerciales son: $R_1 = 10~[k\Omega]$ y $R_2 = 3.9~[k\Omega]$} para el divisor de voltaje propuesto, pudiendo ocupar los 1024 valores que nos da este convertidor.\newline

		\subsection{Suponga que se tiene un ADC que puede opera a una frecuencia de hasta 1kHz. ¿Qué estrategia se recomienda para maximizar la precisión del sensor? ¿Conviene seguir utilizando el promedio simple u otra técnica de filtrado? ¿Es conveniente reducir la frecuencia de muestreo? Justifique su respuesta}
		En el mercado actual existen diferentes tipos de ADC. Estos dispositivos tienen como uno de sus principales compromisos un incremento de la resolución, lo cual se hace a costa de perder frecuencia de trabajo y, normalmente, de incrementar el consumo \cite{lozano2017estudio}.\\ \\
		La función de un convertidor es hacer un ``mapeado'' lineal de una tensión analógica de entrada a una palabra digital de salida, por lo que la curva de transferencia ideal tiene una forma de escalera con todos los escalones iguales alineados en una línea recta. Estos escalones hacen que exista un error de cuantización, ya que cada palabra de salida representa un rango continuo de valores de tensión de entrada \cite{lozano2017estudio}.\\ \\
		Para maximizar la precisión del sensor ADC, se recomienda utilizar los sigma-delta (desarrollados con arquitectura S-D -admite ADC de 24 bits con una resolución p-p de hasta 21.7 bits estables o sin parpadeo-) porque constantemente se obtienen muestras de la entrada analógica y la frecuencia de muestreo es bastante más alta que la banda de interés. También usan conformación de ruido, que empuja a éste fuera de la banda de interés y lo lleva a una región que no esté siendo usada por el proceso de conversión, lo que reduce aún más el ruido en la banda de interés \cite{ADC_sigma}.\\ \\
		Es conveniente utilizar el filtro digital debido a que atenúa cualquier señal fuera de la banda de interés. Este filtro tiene múltiples imágenes en la frecuencia de muestreo; por lo tanto, se requieren algunos filtros externos antialiasing. Sin embargo, debido al sobremuestreo, un filtro RC simple de primer orden es suficiente para casi todas las aplicaciones \cite{ADC_sigma}. \\ \\
		No es conveniente reducir la frecuencia de muestreo, ya que está directamente realacionada con la frecuencia máxima que puede tener la señal a discretizar. Este decremento pudiera producir solapamiento en el espectro (\emph{aliasing}), y con ello hacer que la información contenida en la porción de espectro que quede solapada se pierda; por lo tanto, recuperar la señal analógica original resultaría imposible \cite{bueno2013diseno}.

	\hfill \break %Salto de línea	
	%Agregando fuentes de consulta
	\bibliographystyle{unsrtnat} %abbrnat es parecido a APA
	\bibliography{referencias}
    
\end{document}
