% %% %%%%%%%%%%%%%%%%%%%%%%%%%%%%%%%%%%%%%%%%%%%%%%%%%%%%%%%%%
%
% Materia: Fundamentos de Sistemas Embebidos
% Fecha de creación: DD/MM/2021
% Descripción: Lista de paquetes requeridos
% Hecho por: Brito Segura Angel
%
% %% %%%%%%%%%%%%%%%%%%%%%%%%%%%%%%%%%%%%%%%%%%%%%%%%%%%%%%%%%
% Archivo principal de LaTeX
%!TEX root = ../main.tex
\usepackage[utf8]{inputenc} %Decodificación del documento
\usepackage[T1]{fontenc} %Agregar LATIN1 para todas las lenguas romances aceptadas
\usepackage[spanish,mexico]{babel} %Traducir a español mexicano, opción es-lcroman para minúsculas las letras romanas en numeraciones
\usepackage{fullpage} %Paquete para margen de la página
\usepackage{fancyhdr} %Cambiar pie de página y encabezados
\usepackage{lastpage} %Obtener número de páginas
\usepackage[usenames,dvipsnames]{xcolor} %Tener variedad de colores

%Referencias en el documento
\usepackage{varioref} %ya no se ocupa pero es necesario para los siguientes
\usepackage{hyperref} %habilitar hipervínculos y modificarlos
\usepackage[noabbrev,nameinlink,spanish]{cleveref} %generar referencias (no abreviar la referencia, link en el nombre de la referencia y número, en español)

% Forma recomendada de citar
\usepackage[square, comma, numbers, sort&compress]{natbib} %De acuerdo a la IEEE
%Otra forma de citar
%\usepackage[backend=biber, style=apa]{biblatex} %Importar la bibliografía con su estilo
%\addbibresource{referencias.bib} %Importar el archivo con la bibliografía

\usepackage[inline]{enumitem} %Opciones de espacio y en línea para viñetas

\usepackage{float} %Tablas flotantes (se recomienda así) e imágenes

\usepackage{csquotes} %Realiza diferentes tipos de citas

%Uso de tablas
\usepackage{booktabs} %Tablas elegantes y profesionales
\usepackage{tabularx}
\usepackage{multicol}
\usepackage{multirow} %Muy rara vez se requiere

\usepackage{caption} %Nos da más opciones en los captions

% Manejo de imágenes
\usepackage{graphicx} %Utilizar imágenes
\usepackage{wrapfig} %Permitir imágenes a un lado del texto

\usepackage{ragged2e} %Diferentes tipos de alineado

\usepackage{listings} %Para introducir código fuente
