% %% %%%%%%%%%%%%%%%%%%%%%%%%%%%%%%%%%%%%%%%%%%%%%%%%%%%%%%%%%
%
% Materia: Fundamentos de Sistemas Embebidos
% Fecha de creación: 30/11/2021
% Descripción: Reporte de lectura sobre 
% Hecho por: Brito Segura Angel
%
% %% %%%%%%%%%%%%%%%%%%%%%%%%%%%%%%%%%%%%%%%%%%%%%%%%%%%%%%%%%
% Raíz del proyecto (este archivo)
%!TEX root = ./main.tex
% Archivo de referencias bibliográficas
%!TEX root = ./referencias.bib

\documentclass[letterpaper,12pt]{article} %Tipo de documento

% Paquetes (importación de librería)
% %% %%%%%%%%%%%%%%%%%%%%%%%%%%%%%%%%%%%%%%%%%%%%%%%%%%%%%%%%%
%
% Materia: Fundamentos de Sistemas Embebidos
% Fecha de creación: 30/11/2021
% Descripción: Lista de paquetes requeridos
% Hecho por: Brito Segura Angel
%
% %% %%%%%%%%%%%%%%%%%%%%%%%%%%%%%%%%%%%%%%%%%%%%%%%%%%%%%%%%%
% Archivo principal de LaTeX
%!TEX root = ../main.tex
\usepackage[utf8]{inputenc} %Decodificación del documento
\usepackage[T1]{fontenc} %Agregar LATIN1 para todas las lenguas romances aceptadas
\usepackage[spanish,mexico]{babel} %Traducir a español mexicano
\usepackage{booktabs}
\usepackage{csquotes}
\usepackage{fancyhdr}
\usepackage{geometry}
\usepackage{graphicx}
\usepackage{lastpage}
\usepackage[all]{nowidow}
\usepackage[inline]{enumitem}
\usepackage[usenames,dvipsnames]{xcolor}
\usepackage{varioref}
\usepackage[hidelinks]{hyperref}
\usepackage[noabbrev,nameinlink,spanish]{cleveref}
\usepackage[square, comma, numbers, sort&compress]{natbib}

% Macros (creación de nuevos comandos y entornos)
% %% %%%%%%%%%%%%%%%%%%%%%%%%%%%%%%%%%%%%%%%%%%%%%%%%%%%%%%%%%
%
% Materia: Fundamentos de Sistemas Embebidos
% Fecha de creación: 30/11/2021
% Descripción: Creación de nuevos comandos
% Hecho por: Brito Segura Angel
%
% %% %%%%%%%%%%%%%%%%%%%%%%%%%%%%%%%%%%%%%%%%%%%%%%%%%%%%%%%%%
% Archivo principal de LaTeX
%!TEX root = ../main.tex
\geometry {
	margin=2cm,
	top=3cm
}
\setlength{\parskip}{1em}
\linespread{1.3}
\pagestyle{fancy}
\fancyhf{} % remove everything
\renewcommand{\headrulewidth}{0pt} % remove lines as well
\rfoot{Página~\thepage~de~\pageref{LastPage}}
\rhead{
	Brito Segura Angel \\
	Fundamentos de Sistemas Embebidos --- Grupo 2
}

% Configuraciones del documento (cambios a nivel documento)
% %% %%%%%%%%%%%%%%%%%%%%%%%%%%%%%%%%%%%%%%%%%%%%%%%%%%%%%%%%%
%
% Materia: Fundamentos de Sistemas Embebidos
% Fecha de creación: DD/MM/2021
% Descripción: Configuración de página y documento en general
% Hecho por: Brito Segura Angel
%
% %% %%%%%%%%%%%%%%%%%%%%%%%%%%%%%%%%%%%%%%%%%%%%%%%%%%%%%%%%%
% Archivo principal de LaTeX
%!TEX root = ../main.tex

\makeatletter %Utilizar referencias con @
%Configuración de los hipervínculos y metadatos del PDF
\hypersetup {
	hidelinks,
	colorlinks=true,
	linkcolor=Black, %pone negro a la referencia de página, ver si no afecta en otros lados
	filecolor=OliveGreen, %\href{run:./file.txt}{File.txt}
	urlcolor=RoyalPurple,
	citecolor=Blue,
	pdfauthor={\@author},
	pdftitle={\@title},
%	pdfpagemode=FullScreen,
	pdfsubject={Fundamentos de Sistemas Embebidos},
	pdfkeywords={},
	pdfproducer={SublimeText3 con Makefile (Ubuntu)},
	pdfcreator={pdflatex, rubber}
}
\makeatother %Regresar a la configuración normal

%Cambiar el nombre de la refencia dada: tipo, en sigular, en plural
\crefname{section}{sección}{secciones}
\Crefname{section}{Sección}{Secciones}
%Agregar los que sean necesarios cambiar a los que trae por defecto
\crefname{equation}{ecuación}{ecuaciones}
\Crefname{equation}{Fórmula}{Fórmulas}


\begin{document}
    \noindent{
		\large
		\textbf{Upgrade (2018)}\\[0pt]
		\color{gray} Reporte de película
		\bigskip
	}\\
	Esta película es un claro ejemplo de la industria 4.0 en todo su esplendor, teniendo como su eje central a la Inteligencia Artificial (IA). \\ \\
	En una ciudad desolada donde la IA va de la mano con el desarrollo de la robótica, se hacen patentes varios “errores” que puedan cometer las máquinas, como lo fue el caso de no poder controlar el carro automático en el que iban Grey y Asha Trace y que determina el desarrollo de la historia. En este violento y premeditado accidente, Grey Trace pierde a su esposa Asha y queda tetrapléjico, situación que lo orilla a apoyarse de Eron Keen, innovador tecnológico, que le propone la implantación de un chip para recuperar su movilidad y probar su tecnología. \\ \\
	Este chip llamado STEM está al “servicio” del cerebro de Grey. Poco tiempo después de la operación, este implante tecnológico empieza a hablar con Grey y juntos buscan venganza por la muerte de Asha. En toda la película se observa claramente el contraste de sociedades, con una brecha poblacional en cada lugar que visitaban, cuya implementación de tecnología no iba aparejada del progreso. \\ \\
	A pesar de la ardua, pero fallida labor de la detective Cortez, Trace comprueba que un control absoluto de las calles no implica necesariamente una disminución de hechos criminales; más aún, debe decirse que los delincuentes lo superan y se vuelven aún más poderosos. \\ \\
	Las máquinas no tienen control sobre sí mismas, por lo que siempre buscan el poder a toda costa y querer controlar a la perfección todo lo de su alrededor buscando una superioridad de razas, como se muestra en estos humanos-máquina con los que tiene que pelear Grey durante toda la película. Sorprendentemente, STEM estuvo engañando a Grey haciéndole creer que tenía el control de la situación y sucedió que en esta sed de justicia, Trace fue confiando en este dispositivo, dejando que la máquina se antepusiera a su propia realidad, “gobernándolo”. \\ \\
	Por otra parte, STEM -como cualquier otra creación tecnológica- no está provista de sentimientos ni de emociones, por lo que debe acotarse que en ningún momento dudó en ir eliminando a cada uno de los integrantes de su equipo, solo porque “cometieron errores”.

	\hfill \break	
	Me gustó la frase en que se menciona que es más fácil evadirse de la realidad entrando en un mundo falso/virtual/creado, ya que en éste es menos difícil la vida. Ello me remontó a otro tema muy actual: el problema de las redes sociales y la enajenación que provocan, cuando pretendemos que cubren un vacío existencial y afectivo, orillándonos a crear un mundo fantástico “hecho a la medida” para cada uno de nosotros. \\ \\
	Finalmente, STEM manda a Grey Trace a un eterno sueño en lo que vive lo que él hubiera querido que pasara en la vida real, logrando así el objetivo de la máquina: que la venganza de Grey lo cegara y no pudiera ver las consecuencias y alcances de cada una de sus acciones. Y vuelvo a ello, a que precisamente esto es lo que logran las redes sociales y, en general, la tecnología: tenernos en sus manos y felizmente complacidos de su servicio, creando enajenación y pérdida de la humanidad que nos debería mantener unidos como especie para lograr mejores sociedades; y cuyas máquinas se hayan desprovistas de ello.
\end{document}
