% %% %%%%%%%%%%%%%%%%%%%%%%%%%%%%%%%%%%%%%%%%%%%%%%%%%%%%%%%%%
%
% Materia: Fundamentos de Sistemas Embebidos
% Fecha de creación: 30/11/2021
% Descripción: Reporte de lectura sobre 
% Hecho por: Brito Segura Angel
%
% %% %%%%%%%%%%%%%%%%%%%%%%%%%%%%%%%%%%%%%%%%%%%%%%%%%%%%%%%%%
% Raíz del proyecto (este archivo)
%!TEX root = ./main.tex
% Archivo de referencias bibliográficas
%!TEX root = ./referencias.bib

\documentclass[letterpaper,12pt]{article} %Tipo de documento

% Paquetes (importación de librería)
% %% %%%%%%%%%%%%%%%%%%%%%%%%%%%%%%%%%%%%%%%%%%%%%%%%%%%%%%%%%
%
% Materia: Fundamentos de Sistemas Embebidos
% Fecha de creación: 30/11/2021
% Descripción: Lista de paquetes requeridos
% Hecho por: Brito Segura Angel
%
% %% %%%%%%%%%%%%%%%%%%%%%%%%%%%%%%%%%%%%%%%%%%%%%%%%%%%%%%%%%
% Archivo principal de LaTeX
%!TEX root = ../main.tex
\usepackage[utf8]{inputenc} %Decodificación del documento
\usepackage[T1]{fontenc} %Agregar LATIN1 para todas las lenguas romances aceptadas
\usepackage[spanish,mexico]{babel} %Traducir a español mexicano
\usepackage{booktabs}
\usepackage{csquotes}
\usepackage{fancyhdr}
\usepackage{geometry}
\usepackage{graphicx}
\usepackage{lastpage}
\usepackage[all]{nowidow}
\usepackage[inline]{enumitem}
\usepackage[usenames,dvipsnames]{xcolor}
\usepackage{varioref}
\usepackage[hidelinks]{hyperref}
\usepackage[noabbrev,nameinlink,spanish]{cleveref}
\usepackage[square, comma, numbers, sort&compress]{natbib}

% Macros (creación de nuevos comandos y entornos)
% %% %%%%%%%%%%%%%%%%%%%%%%%%%%%%%%%%%%%%%%%%%%%%%%%%%%%%%%%%%
%
% Materia: Fundamentos de Sistemas Embebidos
% Fecha de creación: 30/11/2021
% Descripción: Creación de nuevos comandos
% Hecho por: Brito Segura Angel
%
% %% %%%%%%%%%%%%%%%%%%%%%%%%%%%%%%%%%%%%%%%%%%%%%%%%%%%%%%%%%
% Archivo principal de LaTeX
%!TEX root = ../main.tex
\geometry {
	margin=2cm,
	top=3cm
}
\setlength{\parskip}{1em}
\linespread{1.3}
\pagestyle{fancy}
\fancyhf{} % remove everything
\renewcommand{\headrulewidth}{0pt} % remove lines as well
\rfoot{Página~\thepage~de~\pageref{LastPage}}
\rhead{
	Brito Segura Angel \\
	Fundamentos de Sistemas Embebidos --- Grupo 2
}

% Configuraciones del documento (cambios a nivel documento)
% %% %%%%%%%%%%%%%%%%%%%%%%%%%%%%%%%%%%%%%%%%%%%%%%%%%%%%%%%%%
%
% Materia: Fundamentos de Sistemas Embebidos
% Fecha de creación: DD/MM/2021
% Descripción: Configuración de página y documento en general
% Hecho por: Brito Segura Angel
%
% %% %%%%%%%%%%%%%%%%%%%%%%%%%%%%%%%%%%%%%%%%%%%%%%%%%%%%%%%%%
% Archivo principal de LaTeX
%!TEX root = ../main.tex

\makeatletter %Utilizar referencias con @
%Configuración de los hipervínculos y metadatos del PDF
\hypersetup {
	hidelinks,
	colorlinks=true,
	linkcolor=Black, %pone negro a la referencia de página, ver si no afecta en otros lados
	filecolor=OliveGreen, %\href{run:./file.txt}{File.txt}
	urlcolor=RoyalPurple,
	citecolor=Blue,
	pdfauthor={\@author},
	pdftitle={\@title},
%	pdfpagemode=FullScreen,
	pdfsubject={Fundamentos de Sistemas Embebidos},
	pdfkeywords={},
	pdfproducer={SublimeText3 con Makefile (Ubuntu)},
	pdfcreator={pdflatex, rubber}
}
\makeatother %Regresar a la configuración normal

%Cambiar el nombre de la refencia dada: tipo, en sigular, en plural
\crefname{section}{sección}{secciones}
\Crefname{section}{Sección}{Secciones}
%Agregar los que sean necesarios cambiar a los que trae por defecto
\crefname{equation}{ecuación}{ecuaciones}
\Crefname{equation}{Fórmula}{Fórmulas}


\begin{document}
    \noindent{
		\large
		\textbf{Why can’t Mexico Make Science Pay Off, Erik Vance}\\[0pt]
		\color{gray} Reporte de lectura
		\bigskip
	}\\
	Este artículo narra acerca de cómo Enrique Reynaud -profesor veterano de biología molecular- pasó de tener todo en 2008, a ver quebrada su empresa de tecnología mexicana en menos de dos años. Al mismo tiempo, esta narración contiene implícitamente los problemas que conllevan crear ciencia en México. \\ \\
	El primer problema que enfrenta es la falta de una cultura de innovación, toda vez que en México no se aprovechan los productos locales -la gran mayoría se exportan al extranjero- ni las tecnologías y empresas emergentes (como lo fue \emph{Biohominis}, que no conocía de su existencia a pesar de ser mexicana). Más aún, debemos tener presente que la UNAM posee la mayor cantidad de estudiantes en todo el hemisferio oeste y que -como es bien sabido- han dado grandes contribuciones al mundo. \\ \\
	Otra problemática está representada por la llamada “guerra contra las drogas” (en la que Estados Unidos constituye el mayor de los consumidores); aunado a otros factores desalentadores en la creación y desarrollo de la ciencia, como lo son la corrupción, las patentes y los nuevos monopolios. El problema va más allá del dinero; a tal grado de que Reynaud comenta que para los inversionistas, tecnología es igual a software y aman a las compañías que dan servicio. \\ \\
	Aunado a lo anterior, existe un gran abuso por parte de las corporaciones que, si bien apuestan por las nuevas ideas, también debe decirse que quieren llevarse el 20\% de beneficio, o bien, ser el dueño de varias acciones de la empresa a apoyar. Comparándonos con Estados Unidos, en California los inversionistas ayudan a reunir y mantener en movimiento las ideas de las empresas que apoyan, entienden la ciencia en su campo y realizan conexiones con laboratorios y dependencias de universidades. En contraste, el gobierno es el peor inversionista y las grandes corporaciones no requieren pagar impuestos directamente, lo que hace que nuevas empresas tengan que financiar prácticamente todo el proyecto y esperar que se genere un retorno de la inversión.

	\hfill \break
	Se tienen a grandes científicos, pero no se tiene un enlace en donde haya la parte técnica en concordancia con las empresas. Somos malinchistas  y no creemos que cosas bien hechas se hagan en México. Además, no se tiene idea de cómo trabajar con la ciencia y los inversionistas no pueden esperar tanto tiempo para que surja una idea. \\ \\
	Pero también se tiene el problema de tener muchos mexicanos en el extranjero en virtud de que la idea de que las universidades pueden apoyar a la industria es nueva (para aquel año 2012), y que a los profesores se les paga con base a sus escritos publicados, careciendo de algún incentivo por patentar o iniciar una nueva empresa. La fuga de cerebros se debe a inversionistas ausentes, burocracia enloquecida y una cultura empresarial anti-riesgos, lo que nos convierte en el país que envía más gente a Estados Unidos de América. \\ \\
	Es preciso destacar que la gran mayoría de las investigaciones son sumamente teóricas; se invierte muy poco en la administración pública para el crecimiento de productos locales, en investigaciones y en desarrollos. \\ \\
	Para el autor, el mayor obstáculo a vencer es la intolerancia al fracaso. Esto se debe a que existen grandes empresas consolidadas, falta de apoyos gubernamentales y monopolios. Esto provoca un crecimiento cultural en que irracionalmente se piensa que siempre se tiene garantizado un retorno de la inversión. \\ \\
	Para el autor del artículo, solo hace falta un soporte para la ciencia y puntualiza que, a pesar de todo lo expuesto, se está produciendo un incremento en historias de éxito en el 2012. Respecto a lo que el autor enfatiza en el sentido de que el dinero no representa un problema, mi punto de vista (en este mundo capitalista y consumista) es que resulta muy necesario; incluso, constituye el primer paso al momento de crear nueva tecnología; la falta de dinero se vuelve una limitante para la creación de nuevas aplicaciones. \\ \\
	Finalmente, quiero comentar que me parece absurda la observación del articulista de que en este año 2012 se tenía al hombre más rico del mundo y, por lo tanto, esto debería de ayudar al desarrollo de la ciencia y tecnología, dado que lamentablemente en este tipo de economía, el hecho de que le vaya muy bien a una persona no influye ni incide directa o indirectamente en el bienestar de los demás.	
\end{document}
