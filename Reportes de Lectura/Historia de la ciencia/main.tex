% %% %%%%%%%%%%%%%%%%%%%%%%%%%%%%%%%%%%%%%%%%%%%%%%%%%%%%%%%%%
%
% Materia: Fundamentos de Sistemas Embebidos
% Fecha de creación: 30/11/2021
% Descripción: Reporte de lectura sobre 
% Hecho por: Brito Segura Angel
%
% %% %%%%%%%%%%%%%%%%%%%%%%%%%%%%%%%%%%%%%%%%%%%%%%%%%%%%%%%%%
% Raíz del proyecto (este archivo)
%!TEX root = ./main.tex
% Archivo de referencias bibliográficas
%!TEX root = ./referencias.bib

\documentclass[letterpaper,12pt]{article} %Tipo de documento

% Paquetes (importación de librería)
% %% %%%%%%%%%%%%%%%%%%%%%%%%%%%%%%%%%%%%%%%%%%%%%%%%%%%%%%%%%
%
% Materia: Fundamentos de Sistemas Embebidos
% Fecha de creación: 10/10/2021
% Descripción: Lista de paquetes requeridos
% Hecho por: Brito Segura Angel
%
% %% %%%%%%%%%%%%%%%%%%%%%%%%%%%%%%%%%%%%%%%%%%%%%%%%%%%%%%%%%
% Archivo principal de LaTeX
%!TEX root = ../main.tex

\usepackage[utf8]{inputenc} %Decodificación del documento
\usepackage[T1]{fontenc} %Agregar LATIN1 para todas las lenguas romances aceptadas
\usepackage[spanish,mexico]{babel} %Traducir a español mexicano
% Forma recomendada de citar
\usepackage[square, comma, numbers, sort&compress]{natbib} %De acuerdo a la IEEE

\usepackage{csquotes} %Diferentes tipos de citas

% Referencias en el documento
\usepackage{varioref} %ya no se ocupa pero es necesario para los siguientes
\usepackage{hyperref} %habilitar hipervínculos y modificarlos
\usepackage[noabbrev,nameinlink,spanish]{cleveref} %generar referencias

\usepackage[colalign]{aligncolsatbottom}  %Para la alineación de las columnas con cierto estilo

% Macros (creación de nuevos comandos y entornos)
% %% %%%%%%%%%%%%%%%%%%%%%%%%%%%%%%%%%%%%%%%%%%%%%%%%%%%%%%%%%
%
% Materia: Fundamentos de Sistemas Embebidos
% Fecha de creación: 30/11/2021
% Descripción: Creación de nuevos comandos
% Hecho por: Brito Segura Angel
%
% %% %%%%%%%%%%%%%%%%%%%%%%%%%%%%%%%%%%%%%%%%%%%%%%%%%%%%%%%%%
% Archivo principal de LaTeX
%!TEX root = ../main.tex
\geometry {
	margin=2cm,
	top=3cm
}
\setlength{\parskip}{1em}
\linespread{1.3}
\pagestyle{fancy}
\fancyhf{} % remove everything
\renewcommand{\headrulewidth}{0pt} % remove lines as well
\rfoot{Página~\thepage~de~\pageref{LastPage}}
\rhead{
	Brito Segura Angel \\
	Fundamentos de Sistemas Embebidos --- Grupo 2
}

% Configuraciones del documento (cambios a nivel documento)
% %% %%%%%%%%%%%%%%%%%%%%%%%%%%%%%%%%%%%%%%%%%%%%%%%%%%%%%%%%%
%
% Materia: Fundamentos de Sistemas Embebidos
% Fecha de creación: 30/11/2021
% Descripción: Configuración de página y documento en general
% Hecho por: Brito Segura Angel
%
% %% %%%%%%%%%%%%%%%%%%%%%%%%%%%%%%%%%%%%%%%%%%%%%%%%%%%%%%%%%
% Archivo principal de LaTeX
%!TEX root = ../main.tex

%Configuración de los hipervínculos y metadatos del PDF
\hypersetup {
	hidelinks,
	colorlinks=true,
	linkcolor=Black, %pone negro a la referencia de página, ver si no afecta en otros lados
	filecolor=OliveGreen, %\href{run:./file.txt}{File.txt}
	urlcolor=RoyalPurple,
	citecolor=Blue,
	pdfauthor={Brito Segura Angel},
	pdftitle={Reporte de lectura - Clean Code},
%	pdfpagemode=FullScreen,
	pdfsubject={Fundamentos de Sistemas Embebidos},
	pdfkeywords={},
	pdfproducer={SublimeText3 con Makefile (Ubuntu)},
	pdfcreator={pdflatex, rubber}
}


\begin{document}
	\hfill \break
    \noindent{
		\large
		\textbf{Historia de la ciencia: ¿cómo se creó la vacuna contra la viruela?, PUCP}\\[0pt]
		\color{gray} Reporte de video
		\bigskip
	}\\
	Este video de Youtube muestra cómo se logra la erradicación de viruela gracias a la vacunación.\\ \\
	En aquellos tiempos que se presentó esta enfermedad altamente contagiosa, no se contaba con ningún tipo de inmunidad y lo más alarmante era que se ocupaba, incluso, como un medio para conseguir un fin determinado; es decir, como un arma. Pero más sorprendente resultó para mí el hecho de que la creación de la vacuna contra la viruela partió de un rumor; así fue que Edward Jenner descubrió cómo las granjeras desarrollaban cierta inmunidad a la viruela bovina, convirtiéndose esta situación en punta de lanza para crearla. \\ \\
	Además, la información transmitida mediante este video hace patente el hecho de que desde entonces el gobierno constituye una pieza clave en la promoción y aplicación de las vacunas. \\ \\
	Toca un tema muy actual a propósito de la pandemia por la que estamos atravesando: las vacunas. No obstante, la problemática en torno a la creación y, desde luego, a la aplicación de las mismas sigue siendo controvertido pese al transcurso del tiempo. En este sentido, la desinformación y manipulación de las noticias, aunado a la falta de credibilidad en torno a la forma en que se crean las vacunas, específicamente en la nueva manera en que se crearon las vacunas contra COVID-19, y más aún, la “prontitud” y el monopolio ejercido, seguirán siendo temas de debate, pero sobretodo de desconfianza de la población respecto de la cual debo decir enfática y convincentemente, que el nivel de alfabetización no es determinante para la aceptación de las vacunas, sino el nivel de transparencia con el que se manejan los procesos que las comprenden.
\end{document}
