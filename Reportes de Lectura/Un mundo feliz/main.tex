% %% %%%%%%%%%%%%%%%%%%%%%%%%%%%%%%%%%%%%%%%%%%%%%%%%%%%%%%%%%
%
% Materia: Fundamentos de Sistemas Embebidos
% Fecha de creación: 30/11/2021
% Descripción: Reporte de lectura sobre 
% Hecho por: Brito Segura Angel
%
% %% %%%%%%%%%%%%%%%%%%%%%%%%%%%%%%%%%%%%%%%%%%%%%%%%%%%%%%%%%
% Raíz del proyecto (este archivo)
%!TEX root = ./main.tex
% Archivo de referencias bibliográficas
%!TEX root = ./referencias.bib

\documentclass[letterpaper,12pt]{article} %Tipo de documento

% Paquetes (importación de librería)
% %% %%%%%%%%%%%%%%%%%%%%%%%%%%%%%%%%%%%%%%%%%%%%%%%%%%%%%%%%%
%
% Materia: Fundamentos de Sistemas Embebidos
% Fecha de creación: 10/10/2021
% Descripción: Lista de paquetes requeridos
% Hecho por: Brito Segura Angel
%
% %% %%%%%%%%%%%%%%%%%%%%%%%%%%%%%%%%%%%%%%%%%%%%%%%%%%%%%%%%%
% Archivo principal de LaTeX
%!TEX root = ../main.tex

\usepackage[utf8]{inputenc} %Decodificación del documento
\usepackage[T1]{fontenc} %Agregar LATIN1 para todas las lenguas romances aceptadas
\usepackage[spanish,mexico]{babel} %Traducir a español mexicano
% Forma recomendada de citar
\usepackage[square, comma, numbers, sort&compress]{natbib} %De acuerdo a la IEEE

\usepackage{csquotes} %Diferentes tipos de citas

% Referencias en el documento
\usepackage{varioref} %ya no se ocupa pero es necesario para los siguientes
\usepackage{hyperref} %habilitar hipervínculos y modificarlos
\usepackage[noabbrev,nameinlink,spanish]{cleveref} %generar referencias

\usepackage[colalign]{aligncolsatbottom}  %Para la alineación de las columnas con cierto estilo

% Macros (creación de nuevos comandos y entornos)
% %% %%%%%%%%%%%%%%%%%%%%%%%%%%%%%%%%%%%%%%%%%%%%%%%%%%%%%%%%%
%
% Materia: Fundamentos de Sistemas Embebidos
% Fecha de creación: 30/11/2021
% Descripción: Creación de nuevos comandos
% Hecho por: Brito Segura Angel
%
% %% %%%%%%%%%%%%%%%%%%%%%%%%%%%%%%%%%%%%%%%%%%%%%%%%%%%%%%%%%
% Archivo principal de LaTeX
%!TEX root = ../main.tex
\geometry {
	margin=2cm,
	top=3cm
}
\setlength{\parskip}{1em}
\linespread{1.3}
\pagestyle{fancy}
\fancyhf{} % remove everything
\renewcommand{\headrulewidth}{0pt} % remove lines as well
\rfoot{Página~\thepage~de~\pageref{LastPage}}
\rhead{
	Brito Segura Angel \\
	Fundamentos de Sistemas Embebidos --- Grupo 2
}

% Configuraciones del documento (cambios a nivel documento)
% %% %%%%%%%%%%%%%%%%%%%%%%%%%%%%%%%%%%%%%%%%%%%%%%%%%%%%%%%%%
%
% Materia: Fundamentos de Sistemas Embebidos
% Fecha de creación: 30/11/2021
% Descripción: Configuración de página y documento en general
% Hecho por: Brito Segura Angel
%
% %% %%%%%%%%%%%%%%%%%%%%%%%%%%%%%%%%%%%%%%%%%%%%%%%%%%%%%%%%%
% Archivo principal de LaTeX
%!TEX root = ../main.tex

%Configuración de los hipervínculos y metadatos del PDF
\hypersetup {
	hidelinks,
	colorlinks=true,
	linkcolor=Black, %pone negro a la referencia de página, ver si no afecta en otros lados
	filecolor=OliveGreen, %\href{run:./file.txt}{File.txt}
	urlcolor=RoyalPurple,
	citecolor=Blue,
	pdfauthor={Brito Segura Angel},
	pdftitle={Reporte de lectura - Clean Code},
%	pdfpagemode=FullScreen,
	pdfsubject={Fundamentos de Sistemas Embebidos},
	pdfkeywords={},
	pdfproducer={SublimeText3 con Makefile (Ubuntu)},
	pdfcreator={pdflatex, rubber}
}


\begin{document}
    \noindent{
		\large
		\textbf{Un mundo feliz, Aldous Huxley}\\[0pt]
		\color{gray} Reporte de lectura
		\bigskip
	}\\
	Este clásico literario narra una “utopía” en un mundo futurista en el que se tiene todo, pero en donde se tienen sociedades cada vez más esclavizadas; incluso puede decirse que conduce a la autodestrucción de la humanidad al estar sumergida en un control total por parte del Estado, provocando en ella una inmensa soledad. \\ \\
	En esta sociedad del futuro, la industrialización ha sido llevada al extremo, a grado tal que los seres humanos son creados a través de ingeniería genética: hombres y mujeres son destinados a una clase social definida y, mediante hipnosis, son convencidos de su rol en la sociedad. Lo extraordinario: no se deja nada al azar. \\ \\
	En este nuevo mundo, la familia y monogamia no existen y las relaciones sexuales promiscuas, como era de esperarse, son alentadas por el propio Estado. Todas las personas consumen Soma (la droga de la “felicidad”) que los tranquiliza y los hace olvidar cualquier problema. Como también era de esperarse, la naturaleza se halla olvidada y menospreciada. \\ \\
	Más aún, un signo de distinción de esta sociedad “vanguardista” es la ignorancia. Esto último es muy importante y cobra absoluta vigencia en nuestros días; así, un pueblo ignorante es más fácil de manipular y controlar por el Estado como se hace en esta sociedad\dots{} del futuro. \\ \\
	Cuando todo parece estar debidamente controlado, Bernard Marx se siente insatisfecho por el rol y casta asignada, así como con la sociedad en la que vive, por lo que decide viajar a Nuevo México para conocer personas que viven como se hacía en la antigüedad. \\ \\
	En contraste entran dos personajes más: Lenina Crowne, que sigue las reglas del Estado sin problemas; y John El Salvaje, que es un joven criado en la naturaleza. Esta combinación de personajes de la novela permite observar los distintos puntos de vista en una misma situación y tiempo; lo cual, trasladándolo a la actualidad, puede palparse por el simple hecho de que, pese a coincidir en un misma época y lugar geográfico, cada persona está provista de distintas perspectivas inherentes al estado en que se encuentra el mundo, conforme a sus experiencias, vivencias y entorno o ambiente en donde se desarrollan.

	\hfill \break
	Si bien, al leerlo se me hizo una lectura algo pesada y con falta de “ritmo”, este mundo futurista refleja hacia dónde va la humanidad si no se toma conciencia de ello: sociedades completamente controladas, manipuladas al antojo de grande empresarios y políticos y, lo más preocupante, un desprecio y falta de interés en la naturaleza que afecta directamente el mundo en el que vivimos. Esta novela nos conduce a una sociedad -no muy distinta a la actual- que requiere de una droga (red social, videojuego, mundo virtual, solo por mencionar algunos ejemplos en nuestra sociedad) para poder olvidar todo lo malo y, lo más desesperanzador: que nadie quiere tomar acción para cambiarla.
\end{document}
