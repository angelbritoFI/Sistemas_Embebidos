% %% %%%%%%%%%%%%%%%%%%%%%%%%%%%%%%%%%%%%%%%%%%%%%%%%%%%%%%%%%
%
% Materia: Fundamentos de Sistemas Embebidos
% Fecha de creación: 30/11/2021
% Descripción: Reporte de lectura sobre 
% Hecho por: Brito Segura Angel
%
% %% %%%%%%%%%%%%%%%%%%%%%%%%%%%%%%%%%%%%%%%%%%%%%%%%%%%%%%%%%
% Raíz del proyecto (este archivo)
%!TEX root = ./main.tex
% Archivo de referencias bibliográficas
%!TEX root = ./referencias.bib

\documentclass[letterpaper,12pt]{article} %Tipo de documento

% Paquetes (importación de librería)
% %% %%%%%%%%%%%%%%%%%%%%%%%%%%%%%%%%%%%%%%%%%%%%%%%%%%%%%%%%%
%
% Materia: Fundamentos de Sistemas Embebidos
% Fecha de creación: 10/10/2021
% Descripción: Lista de paquetes requeridos
% Hecho por: Brito Segura Angel
%
% %% %%%%%%%%%%%%%%%%%%%%%%%%%%%%%%%%%%%%%%%%%%%%%%%%%%%%%%%%%
% Archivo principal de LaTeX
%!TEX root = ../main.tex

\usepackage[utf8]{inputenc} %Decodificación del documento
\usepackage[T1]{fontenc} %Agregar LATIN1 para todas las lenguas romances aceptadas
\usepackage[spanish,mexico]{babel} %Traducir a español mexicano
% Forma recomendada de citar
\usepackage[square, comma, numbers, sort&compress]{natbib} %De acuerdo a la IEEE

\usepackage{csquotes} %Diferentes tipos de citas

% Referencias en el documento
\usepackage{varioref} %ya no se ocupa pero es necesario para los siguientes
\usepackage{hyperref} %habilitar hipervínculos y modificarlos
\usepackage[noabbrev,nameinlink,spanish]{cleveref} %generar referencias

\usepackage[colalign]{aligncolsatbottom}  %Para la alineación de las columnas con cierto estilo

% Macros (creación de nuevos comandos y entornos)
% %% %%%%%%%%%%%%%%%%%%%%%%%%%%%%%%%%%%%%%%%%%%%%%%%%%%%%%%%%%
%
% Materia: Fundamentos de Sistemas Embebidos
% Fecha de creación: 30/11/2021
% Descripción: Creación de nuevos comandos
% Hecho por: Brito Segura Angel
%
% %% %%%%%%%%%%%%%%%%%%%%%%%%%%%%%%%%%%%%%%%%%%%%%%%%%%%%%%%%%
% Archivo principal de LaTeX
%!TEX root = ../main.tex
\geometry {
	margin=2cm,
	top=3cm
}
\setlength{\parskip}{1em}
\linespread{1.3}
\pagestyle{fancy}
\fancyhf{} % remove everything
\renewcommand{\headrulewidth}{0pt} % remove lines as well
\rfoot{Página~\thepage~de~\pageref{LastPage}}
\rhead{
	Brito Segura Angel \\
	Fundamentos de Sistemas Embebidos --- Grupo 2
}

% Configuraciones del documento (cambios a nivel documento)
% %% %%%%%%%%%%%%%%%%%%%%%%%%%%%%%%%%%%%%%%%%%%%%%%%%%%%%%%%%%
%
% Materia: Fundamentos de Sistemas Embebidos
% Fecha de creación: 30/11/2021
% Descripción: Configuración de página y documento en general
% Hecho por: Brito Segura Angel
%
% %% %%%%%%%%%%%%%%%%%%%%%%%%%%%%%%%%%%%%%%%%%%%%%%%%%%%%%%%%%
% Archivo principal de LaTeX
%!TEX root = ../main.tex

%Configuración de los hipervínculos y metadatos del PDF
\hypersetup {
	hidelinks,
	colorlinks=true,
	linkcolor=Black, %pone negro a la referencia de página, ver si no afecta en otros lados
	filecolor=OliveGreen, %\href{run:./file.txt}{File.txt}
	urlcolor=RoyalPurple,
	citecolor=Blue,
	pdfauthor={Brito Segura Angel},
	pdftitle={Reporte de lectura - Clean Code},
%	pdfpagemode=FullScreen,
	pdfsubject={Fundamentos de Sistemas Embebidos},
	pdfkeywords={},
	pdfproducer={SublimeText3 con Makefile (Ubuntu)},
	pdfcreator={pdflatex, rubber}
}


\begin{document}
    \noindent{
		\large
		\textbf{8-Bits of Advice for New Programmers, javidx9}\\[0pt]
		\color{gray} Reporte del video
		\bigskip
	}\\
	Este video de Youtube es muy bueno e interesante por tratarse de consejos para los programadores que apenas están iniciándose en sus desarrollos. Me pareció excelente que hiciera especial énfasis en el punto de que la programación es entender la lógica y naturaleza del código (ya que lo demás se trata de sintaxis).\\ \\
	El primer consejo -importantísimo- es que debe partirse por entender lo básico y poco a poco ir subiendo de nivel, tal y como sucede en el ámbito educativo, donde en la escuela primero se nos enseña lo más básico y poco a poco estas enseñanzas van incrementando su grado de dificultad. \\ \\
	Respecto al segundo consejo difiero de alguna manera con el youtuber ya que, si bien es cierto que es muy bueno compilar uno solo para entender el proceso completo, no comparto el hecho de ocupar entornos completos de desarrollo (IDE) porque ello conduce muchas veces a que el programador no aprenda la lógica debido al autocompletado de código y otras funciones en estos entornos. Considero que lo mejor es iniciar con un simple editor de texto y en línea de comandos entender todo el proceso. Por otra parte, menciona que lo mejor es correr el código línea por línea y ver qué hace, lo cual no me parece acertado en virtud de que atrasa la mente creativa del nuevo programador y lo hace desconfiar de cada paso que está haciendo. \\ \\
	El tercer consejo me parece muy bueno, ya que acercar a los programadores en aplicaciones reales o en algo que uno pudiera ocupar hace que no se pierda el interés por aprender; incluso, ello hace que se vuelva una necesidad el que todo salga bien. \\ \\
	El siguiente consejo resulta obligado recomendarlo: “la mejor manera de aprender algo es practicándolo”. Además, comparto una muy buena idea: la de que los programas sean cortos, sencillos y con una sola función para que, de primera instancia, al nuevo programador le puede servir para que él mismo tenga su propia “librería”, que pueda consultar y para que sepa exactamente que hace cada uno. De igual manera, es bueno que al programador se le introduzca desde sus inicios a modularizar ya que esto es muy ocupado hoy en día y nos sirve para tener mejor ordenado el código; inclusive, sirve para reutilizarlo en la resolución de problemas mayores.\\ \\
	El quinto consejo que se da en este video es algo a lo que siempre se le induce al nuevo programador: cortar y pegar de internet para entender algo. Esto es muy malo porque con ello solo se está mirando hacia la sintaxis, sin entender el proceso de creación de la función correspondiente o la solución a un problema. \\ \\
	Adicionalmente, comenta que es bueno manejar una convención de nombres con un buen nombrado de variables para que el código sea lo más claro posible. Esto último, en mi punto de vista, contradice lo dicho en el penúltimo consejo, ya que si bien podemos fallar, tronar o incluso dañar el entorno de trabajo, aconseja que a los nuevos programadores no debe importarles no seguir las buenas prácticas; lo que haría que las convenciones y el nombrado tampoco importaran. Considero que, si bien no es bueno abrumar a los nuevos programadores con estándares, normas, reglas y otras cosas, debería guiárseles con pequeños tips para que se ocupen las buenas prácticas. Esto provocaría que, con el paso del tiempo, se vuelva un hábito y así que cuando se trate de aplicaciones reales ya se tengan estos antecedentes que les permitan trabajar en equipo, mucho más rápido y eficientar los resultados obtenidos. \\ \\
	El sexto consejo es muy contradictorio, ya que el youtuber dice que los programadores no deben escuchar gente de Internet -lo cual concuerdo completamente con él-, pero su video y estos consejos están en esta red. \\ \\
	Finalmente, comparto su idea de que la mejor manera de aprender es cometiendo errores. Muchas veces venimos de culturas o costumbres que buscan la perfección en todo lo que se hace; sin embargo, esto frena el aprendizaje ya que si uno no comete errores no es capaz de observar sus áreas de oportunidad y, consecuentemente, no logra crecer profesionalmente. \\ \\
	Como comentario final, quiero agregar que me hubiera gustado haber visto este video cuando comencé a programar, aunque debo reconocer que actualmente sigue siendo de gran ayuda para recordar las buenas prácticas y ser un mejor programador.
\end{document}
