% %% %%%%%%%%%%%%%%%%%%%%%%%%%%%%%%%%%%%%%%%%%%%%%%%%%%%%%%%%%
%
% Materia: Fundamentos de Sistemas Embebidos
% Fecha de creación: 30/11/2021
% Descripción: Reporte de lectura sobre 
% Hecho por: Brito Segura Angel
%
% %% %%%%%%%%%%%%%%%%%%%%%%%%%%%%%%%%%%%%%%%%%%%%%%%%%%%%%%%%%
% Raíz del proyecto (este archivo)
%!TEX root = ./main.tex
% Archivo de referencias bibliográficas
%!TEX root = ./referencias.bib

\documentclass[letterpaper,12pt]{article} %Tipo de documento

% Paquetes (importación de librería)
% %% %%%%%%%%%%%%%%%%%%%%%%%%%%%%%%%%%%%%%%%%%%%%%%%%%%%%%%%%%
%
% Materia: Fundamentos de Sistemas Embebidos
% Fecha de creación: 10/10/2021
% Descripción: Lista de paquetes requeridos
% Hecho por: Brito Segura Angel
%
% %% %%%%%%%%%%%%%%%%%%%%%%%%%%%%%%%%%%%%%%%%%%%%%%%%%%%%%%%%%
% Archivo principal de LaTeX
%!TEX root = ../main.tex

\usepackage[utf8]{inputenc} %Decodificación del documento
\usepackage[T1]{fontenc} %Agregar LATIN1 para todas las lenguas romances aceptadas
\usepackage[spanish,mexico]{babel} %Traducir a español mexicano
% Forma recomendada de citar
\usepackage[square, comma, numbers, sort&compress]{natbib} %De acuerdo a la IEEE

\usepackage{csquotes} %Diferentes tipos de citas

% Referencias en el documento
\usepackage{varioref} %ya no se ocupa pero es necesario para los siguientes
\usepackage{hyperref} %habilitar hipervínculos y modificarlos
\usepackage[noabbrev,nameinlink,spanish]{cleveref} %generar referencias

\usepackage[colalign]{aligncolsatbottom}  %Para la alineación de las columnas con cierto estilo

% Macros (creación de nuevos comandos y entornos)
% %% %%%%%%%%%%%%%%%%%%%%%%%%%%%%%%%%%%%%%%%%%%%%%%%%%%%%%%%%%
%
% Materia: Fundamentos de Sistemas Embebidos
% Fecha de creación: 30/11/2021
% Descripción: Creación de nuevos comandos
% Hecho por: Brito Segura Angel
%
% %% %%%%%%%%%%%%%%%%%%%%%%%%%%%%%%%%%%%%%%%%%%%%%%%%%%%%%%%%%
% Archivo principal de LaTeX
%!TEX root = ../main.tex
\geometry {
	margin=2cm,
	top=3cm
}
\setlength{\parskip}{1em}
\linespread{1.3}
\pagestyle{fancy}
\fancyhf{} % remove everything
\renewcommand{\headrulewidth}{0pt} % remove lines as well
\rfoot{Página~\thepage~de~\pageref{LastPage}}
\rhead{
	Brito Segura Angel \\
	Fundamentos de Sistemas Embebidos --- Grupo 2
}

% Configuraciones del documento (cambios a nivel documento)
% %% %%%%%%%%%%%%%%%%%%%%%%%%%%%%%%%%%%%%%%%%%%%%%%%%%%%%%%%%%
%
% Materia: Fundamentos de Sistemas Embebidos
% Fecha de creación: 30/11/2021
% Descripción: Configuración de página y documento en general
% Hecho por: Brito Segura Angel
%
% %% %%%%%%%%%%%%%%%%%%%%%%%%%%%%%%%%%%%%%%%%%%%%%%%%%%%%%%%%%
% Archivo principal de LaTeX
%!TEX root = ../main.tex

%Configuración de los hipervínculos y metadatos del PDF
\hypersetup {
	hidelinks,
	colorlinks=true,
	linkcolor=Black, %pone negro a la referencia de página, ver si no afecta en otros lados
	filecolor=OliveGreen, %\href{run:./file.txt}{File.txt}
	urlcolor=RoyalPurple,
	citecolor=Blue,
	pdfauthor={Brito Segura Angel},
	pdftitle={Reporte de lectura - Clean Code},
%	pdfpagemode=FullScreen,
	pdfsubject={Fundamentos de Sistemas Embebidos},
	pdfkeywords={},
	pdfproducer={SublimeText3 con Makefile (Ubuntu)},
	pdfcreator={pdflatex, rubber}
}


\begin{document}
    \noindent{
		\large
		\textbf{Hidden Figures (2016)}\\[0pt]
		\color{gray} Reporte de película
		\bigskip
	}\\
	Esta película trata sobre tres mujeres matemáticas que ayudaron a lanzar al espacio al astronauta John Glenn en el año 1961. Cuenta la historia de la matemática Katherine Johnson, quien calculó las trayectorias de lanzamiento y aterrizaje que hicieron posible el vuelo del Proyecto Mercurio y la llegada a la luna del Apolo 11 en 1969. Paralelamente, expone la hazaña de sus colegas, Dorothy Vaughn y Mary Jackson; todas ellas gente de color que tuvieron que enfrentarse a una injusta cotidianidad en esa época -y en la actual-, al mantenerlas invisibilizadas y menospreciadas. \\ \\
	Me impresionó ver cómo hasta los protocolos de la NASA prohibían la presencia de estas tres mujeres brillantes en juntas laborales y que sus compañeros -blancos- de oficina asumían la autoría de sus trabajos. Pero más denigrante para mí resultó el hecho de que ellas tenían que ir hasta el otro edificio para realizar sus necesidades fisiológicas. \\ \\
	En esta lucha por mantenerse y aguantar las intensas exigencias impuestas por su raza -y más aún, por su sexo-, nos deja ver la parte humana de cada una de ellas: mujeres que se enamoran, sufren, viven por sus hijos y luchan para alcanzar a ser alguien en un mundo estereotipado, dominado por gente blanca. \\ \\
	Esta película que desdeña el trabajo de la mujer y que resulta reprochable por la discriminación racial, me pareció excepcional precisamente por lo contrario, ya que bajo mi óptica empodera el trabajo de las mujeres y es incluyente, en una industria, la cinematográfica, que aún en nuestros días se rehúsa a esta equidad de género, tanto en el número de actores blancos, así como en contenidos dotados de supremacía racial. \\ \\
	Es inadmisible que en pleno siglo XXI sigamos “consumiendo” confrontación y división por cuestiones de raza, sexo o condición social. Apelo porque estas muestras fílmicas de los típicos personajes (hombre blanco que no puede aceptar alguien de raza negra y, menos aún, una mujer que sea superior a él) y la falta de apoyo entre mujeres, nos permitan reflexionar sobre la importancia de aceptarnos y valorarnos como individuos\dots{} sí, pero también por nuestras capacidades y competencias, dejando de lado cualquier tópico o cliché; y nunca más fomentemos el rechazo por el hecho de que otra persona sea distinta a nosotros.
\end{document}
