% %% %%%%%%%%%%%%%%%%%%%%%%%%%%%%%%%%%%%%%%%%%%%%%%%%%%%%%%%%%
%
% Materia: Fundamentos de Sistemas Embebidos
% Fecha de creación: 30/11/2021
% Descripción: Reporte de lectura sobre 
% Hecho por: Brito Segura Angel
%
% %% %%%%%%%%%%%%%%%%%%%%%%%%%%%%%%%%%%%%%%%%%%%%%%%%%%%%%%%%%
% Raíz del proyecto (este archivo)
%!TEX root = ./main.tex
% Archivo de referencias bibliográficas
%!TEX root = ./referencias.bib

\documentclass[letterpaper,12pt]{article} %Tipo de documento

% Paquetes (importación de librería)
% %% %%%%%%%%%%%%%%%%%%%%%%%%%%%%%%%%%%%%%%%%%%%%%%%%%%%%%%%%%
%
% Materia: Fundamentos de Sistemas Embebidos
% Fecha de creación: 30/11/2021
% Descripción: Lista de paquetes requeridos
% Hecho por: Brito Segura Angel
%
% %% %%%%%%%%%%%%%%%%%%%%%%%%%%%%%%%%%%%%%%%%%%%%%%%%%%%%%%%%%
% Archivo principal de LaTeX
%!TEX root = ../main.tex
\usepackage[utf8]{inputenc} %Decodificación del documento
\usepackage[T1]{fontenc} %Agregar LATIN1 para todas las lenguas romances aceptadas
\usepackage[spanish,mexico]{babel} %Traducir a español mexicano
\usepackage{booktabs}
\usepackage{csquotes}
\usepackage{fancyhdr}
\usepackage{geometry}
\usepackage{graphicx}
\usepackage{lastpage}
\usepackage[all]{nowidow}
\usepackage[inline]{enumitem}
\usepackage[usenames,dvipsnames]{xcolor}
\usepackage{varioref}
\usepackage[hidelinks]{hyperref}
\usepackage[noabbrev,nameinlink,spanish]{cleveref}
\usepackage[square, comma, numbers, sort&compress]{natbib}

% Macros (creación de nuevos comandos y entornos)
% %% %%%%%%%%%%%%%%%%%%%%%%%%%%%%%%%%%%%%%%%%%%%%%%%%%%%%%%%%%
%
% Materia: Fundamentos de Sistemas Embebidos
% Fecha de creación: 30/11/2021
% Descripción: Creación de nuevos comandos
% Hecho por: Brito Segura Angel
%
% %% %%%%%%%%%%%%%%%%%%%%%%%%%%%%%%%%%%%%%%%%%%%%%%%%%%%%%%%%%
% Archivo principal de LaTeX
%!TEX root = ../main.tex
\geometry {
	margin=2cm,
	top=3cm
}
\setlength{\parskip}{1em}
\linespread{1.3}
\pagestyle{fancy}
\fancyhf{} % remove everything
\renewcommand{\headrulewidth}{0pt} % remove lines as well
\rfoot{Página~\thepage~de~\pageref{LastPage}}
\rhead{
	Brito Segura Angel \\
	Fundamentos de Sistemas Embebidos --- Grupo 2
}

% Configuraciones del documento (cambios a nivel documento)
% %% %%%%%%%%%%%%%%%%%%%%%%%%%%%%%%%%%%%%%%%%%%%%%%%%%%%%%%%%%
%
% Materia: Fundamentos de Sistemas Embebidos
% Fecha de creación: DD/MM/2021
% Descripción: Configuración de página y documento en general
% Hecho por: Brito Segura Angel
%
% %% %%%%%%%%%%%%%%%%%%%%%%%%%%%%%%%%%%%%%%%%%%%%%%%%%%%%%%%%%
% Archivo principal de LaTeX
%!TEX root = ../main.tex

\makeatletter %Utilizar referencias con @
%Configuración de los hipervínculos y metadatos del PDF
\hypersetup {
	hidelinks,
	colorlinks=true,
	linkcolor=Black, %pone negro a la referencia de página, ver si no afecta en otros lados
	filecolor=OliveGreen, %\href{run:./file.txt}{File.txt}
	urlcolor=RoyalPurple,
	citecolor=Blue,
	pdfauthor={\@author},
	pdftitle={\@title},
%	pdfpagemode=FullScreen,
	pdfsubject={Fundamentos de Sistemas Embebidos},
	pdfkeywords={},
	pdfproducer={SublimeText3 con Makefile (Ubuntu)},
	pdfcreator={pdflatex, rubber}
}
\makeatother %Regresar a la configuración normal

%Cambiar el nombre de la refencia dada: tipo, en sigular, en plural
\crefname{section}{sección}{secciones}
\Crefname{section}{Sección}{Secciones}
%Agregar los que sean necesarios cambiar a los que trae por defecto
\crefname{equation}{ecuación}{ecuaciones}
\Crefname{equation}{Fórmula}{Fórmulas}


\begin{document}
    \noindent{
		\large
		\textbf{Ex Machina (2014)}\\[0pt]
		\color{gray} Reporte de película
		\bigskip
	}\\
	Esta película trata sobre la Inteligencia Artificial (IA) aplicada específicamente en la creación de humanoides, así como de su impacto al proveerlos de conciencia y habla.\\ \\
	Nathan, un gran erudito cuya empresa cuenta con el buscador número 1 de Internet, selecciona a un joven programador, Caleb, para probar su creación: Ava. Para ello, entra a una especie de centro de investigación en donde Nathan es quien se encarga exclusivamente de operar y controlar todas las actividades inherentes a esta revolución digital. Nathan cuenta con un mapa de lo que piensa la gente gracias a la vigilancia permanente de cada movimiento que realizan en sus dispositivos, situación por la cual maneja recursos ilimitados para la creación de sus máquinas. Consecuentemente, posee un control excesivo y desvirtuado sobre la información que maneja; basta recordar que crea la cara de Ava para satisfacer los “deseos” -pornográficos- sexuales de Caleb; pero además, este control juega en su contra, en un momento determinado, por la inminente supremacía que presentan estos humanoides dotados de IA sobre su propio creador.\\ \\
	Me llamó mucho la atención el hecho de tener encerrada a Ava porque, en definitiva, hace patente lo incierto de los alcances de la IA; más aún, el miedo que les generaba pensar que, escapando, las máquinas pudieran dominar a la humanidad. Al mismo tiempo, resulta trascendental el contraste entre simulación (creación de un mundo robótico) contra la realidad (el bosque y naturaleza en el que se halla inmerso el centro de investigación).\\ \\
	Caleb refiere a Nathan que para conseguir que una máquina tenga conciencia requiere sentir, es decir, experimentar emociones; pero es Ava quien sorprendentemente con sus dibujos cotidianos va “transformando” a Caleb, logrando entrar a su mente para obtener información, manipulándolo a través del control de sus emociones. Nathan es tan consciente de lo importante que resultan las emociones como método de control, que la elección de Caleb no fue fortuita; por el contrario, pensó muy bien el perfil que deseaba: un programador, el más talentoso de su compañía, subsumido en una inmensa soledad. Gracias a la seducción y empatía que consigue Ava sobre Caleb, ella logra salir de “aquella prisión”, mata a su creador y continúa su “vida” como uno más de nosotros, en una ciudad de la que queda completamente embelesada.\\ \\
	Si bien esta película es de ciencia ficción, llego a la convicción de que cuando confiamos demasiado -incluso ciegamente- en algoritmos, máquinas y programas, el ser humano puede perder lo que realmente lo hace ser tal: su humanidad; es decir, esta capacidad que poseemos para sentir afecto, comprensión y solidaridad hacia las demás personas.\\ \\
	En esta película se muestran posibles (y algunos, actuales) alcances de la tecnología; específicamente sobre la Inteligencia Artificial que, desde su creación, ha causado mucha controversia por lo que pudiese llegar a suceder, dado que hasta este momento no existe sistema perfecto creado por humanos, pero sí aquellos que dependiendo de determinados factores pueden alterar su funcionamiento; por ejemplo, una simple falla de energía. Por otra parte en Nathan y Caleb se muestran las patologías y trastornos que aquejan a la humanidad. Permitir que un empresario ambicioso y egocéntrico manipule y controle todo a su beneficio no es correcto; sin embargo, en la vida real se crean máquinas, algoritmos, programas, entre otras tecnologías, para placer y fines individuales cuando se debería de buscar el bien común.\\ \\
	En mi opinión, esta película está llena de contrastes; principalmente y el más notorio -como ya lo mencioné- el de la tecnología y naturaleza, ya que a pesar de que se tenía todo y lo mejor en el centro de investigación (pero que no obstante ello en un momento de la película resulta abrumador), constantemente Nathan y Caleb salían a disfrutar de la naturaleza, la cual considero que era un medio de escape a todo este dominio de la IA y, desde luego, sus creaciones.\\ \\
	Mientras los avances de la tecnología se den a conocer a capricho, se mantengan ciertos aspectos de la misma con carácter de confidencialidad y no se den a conocer cómo se crean, estaremos condenados a vivir a merced de los grupos más poderosos, no solo económicamente hablando, sino también de los “poseedores” de toda esta nueva industria 4.0.
\end{document}
