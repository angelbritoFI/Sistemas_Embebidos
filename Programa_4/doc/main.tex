% %% %%%%%%%%%%%%%%%%%%%%%%%%%%%%%%%%%%%%%%%%%%%%%%%%%%%%%%%%%
%
% Materia: Fundamentos de Sistemas Embebidos
% Fecha de creación: 15/11/2021
% Descripción: Cuestionario del último programa de la materia
% Hecho por: Brito Segura Angel
%
% %% %%%%%%%%%%%%%%%%%%%%%%%%%%%%%%%%%%%%%%%%%%%%%%%%%%%%%%%%%
% Raíz del proyecto (este archivo)
%!TEX root = ./main.tex
% Archivo de referencias bibliográficas
%!TEX root = ./referencias.bib

                % Tamaño de hoja y letra (11pt o 12pt -10pt por default-)
\documentclass[letterpaper,10.5pt]{article} %Tipo de documento

% Paquetes (importación de librería)
% %% %%%%%%%%%%%%%%%%%%%%%%%%%%%%%%%%%%%%%%%%%%%%%%%%%%%%%%%%%
%
% Materia: Fundamentos de Sistemas Embebidos
% Fecha de creación: 10/10/2021
% Descripción: Lista de paquetes requeridos
% Hecho por: Brito Segura Angel
%
% %% %%%%%%%%%%%%%%%%%%%%%%%%%%%%%%%%%%%%%%%%%%%%%%%%%%%%%%%%%
% Archivo principal de LaTeX
%!TEX root = ../main.tex

\usepackage[utf8]{inputenc} %Decodificación del documento
\usepackage[T1]{fontenc} %Agregar LATIN1 para todas las lenguas romances aceptadas
\usepackage[spanish,mexico]{babel} %Traducir a español mexicano
% Forma recomendada de citar
\usepackage[square, comma, numbers, sort&compress]{natbib} %De acuerdo a la IEEE

\usepackage{csquotes} %Diferentes tipos de citas

% Referencias en el documento
\usepackage{varioref} %ya no se ocupa pero es necesario para los siguientes
\usepackage{hyperref} %habilitar hipervínculos y modificarlos
\usepackage[noabbrev,nameinlink,spanish]{cleveref} %generar referencias

\usepackage[colalign]{aligncolsatbottom}  %Para la alineación de las columnas con cierto estilo

% Macros (creación de nuevos comandos y entornos)
% %% %%%%%%%%%%%%%%%%%%%%%%%%%%%%%%%%%%%%%%%%%%%%%%%%%%%%%%%%%
%
% Materia: Fundamentos de Sistemas Embebidos
% Fecha de creación: 30/11/2021
% Descripción: Creación de nuevos comandos
% Hecho por: Brito Segura Angel
%
% %% %%%%%%%%%%%%%%%%%%%%%%%%%%%%%%%%%%%%%%%%%%%%%%%%%%%%%%%%%
% Archivo principal de LaTeX
%!TEX root = ../main.tex
\geometry {
	margin=2cm,
	top=3cm
}
\setlength{\parskip}{1em}
\linespread{1.3}
\pagestyle{fancy}
\fancyhf{} % remove everything
\renewcommand{\headrulewidth}{0pt} % remove lines as well
\rfoot{Página~\thepage~de~\pageref{LastPage}}
\rhead{
	Brito Segura Angel \\
	Fundamentos de Sistemas Embebidos --- Grupo 2
}


\title{Programa 4}
\author{\textbf{Brito Segura Angel}}
\date{Fecha de entrega: \today}

%Configuraciones del documento (cambios a nivel documento)
% %% %%%%%%%%%%%%%%%%%%%%%%%%%%%%%%%%%%%%%%%%%%%%%%%%%%%%%%%%%
%
% Materia: Fundamentos de Sistemas Embebidos
% Fecha de creación: 30/11/2021
% Descripción: Configuración de página y documento en general
% Hecho por: Brito Segura Angel
%
% %% %%%%%%%%%%%%%%%%%%%%%%%%%%%%%%%%%%%%%%%%%%%%%%%%%%%%%%%%%
% Archivo principal de LaTeX
%!TEX root = ../main.tex

%Configuración de los hipervínculos y metadatos del PDF
\hypersetup {
	hidelinks,
	colorlinks=true,
	linkcolor=Black, %pone negro a la referencia de página, ver si no afecta en otros lados
	filecolor=OliveGreen, %\href{run:./file.txt}{File.txt}
	urlcolor=RoyalPurple,
	citecolor=Blue,
	pdfauthor={Brito Segura Angel},
	pdftitle={Reporte de lectura - Clean Code},
%	pdfpagemode=FullScreen,
	pdfsubject={Fundamentos de Sistemas Embebidos},
	pdfkeywords={},
	pdfproducer={SublimeText3 con Makefile (Ubuntu)},
	pdfcreator={pdflatex, rubber}
}


\begin{document}
    \pagestyle{fancy} %Estilo de página
    \maketitle %Poner el título, autor y fecha al documento
    
    %Para solo una página, descomentar estas líneas y quitar TODO lo de fancy
    %\pagestyle{empty} %Quitar paginación
    %\maketitle
    %\thispagestyle{empty}
    
    \section{Cuestionario}
	
	\subsection{Presente la tabla o función de aproximación utilizada para la obtención del tiempo de desfase en función del percentil de potencia $\tau$($P_f$) para una línea de AC que opera a 50Hz}
	\noindent
	A continuación se muestra la tabla para interpolación con incrementos de 5\% y alimentación de AC para el \emph{dimmer} implementado a 50 [Hz]:
		\begin{table}[H] %Opción H para que se muestre donde está
			\centering %Centrar tabla
			\begin{tabular}{c c} %geometría de la tabla (izquierda, centrado, derecho)
				% | genera línea en el contenido de la tabla		
				\toprule %Mostrar línea al inicio de la tabla
				% Encabezados de las columnas {cuantas columnas abarca, alineación y texto}
				\multicolumn{1}{c}{\bfseries Factor de potencia [\%]} &
				\multicolumn{1}{c}{\bfseries Tiempo de disparo [ms]} \\
				\midrule %Poner línea entre encabezados e información de las filas
				% & -> separador de columnas, \\ -> separardor de filas
				0 & 10 \\
				5 & 8.564 \\
				10 & 7.952 \\
				15 & 7.468 \\
				20 & 7.048 \\
				25 & 6.6666 \\
				30 & 6.31 \\
				35 & 5.97 \\
				40 & 5.641 \\
				45 & 5.319 \\
				50 & 5 \\
				55 & 4.681 \\
				60 & 4.359 \\
				65 & 4.03 \\
				70 & 3.69 \\
				75 & 3.3333 \\
				80 & 2.952 \\
				85 & 2.532 \\
				90 & 2.048 \\
				95 & 1.435 \\
				100 & 0 \\
				\bottomrule %Poner línea al finalizar tabla
			\end{tabular}
			\caption{Relación entre factor de potencia y tiempo de disparo de TRIAC} %Nombre de la tabla
			\label{tbl:pf/tau} %TODO caption lleva su etiqueta
		\end{table}
	\subsection{¿Es posible utilizar este método de atenuación con otro tipo de lámparas (ej. Fluorescente o LED)? La detección de cruce por cero, ¿puede utilizarse para modular la potencia de cargas inductivas?}
	El interruptor de perilla (\emph{dimmer}) se usa en la iluminación tradicional para tener un control de atenuación en fuentes de luz como lámparas incandescentes, ya que este tipo de lámparas consumen excesiva energía. Este método de atenuación no es recomendable para otro tipo lámparas porque puede dañarlas (incluso, sus propias especificaciones indican que no se ocupen para este própósito). Con el aumento en el uso de lámparas LED, el uso de estos \emph{dimmers} es menor y cada vez son más rechazados en la industria \cite{espinel2018modelacion}.\\ \\
	La detección de cruce por cero se puede utilizar para modular la potencia de cargas inductivas \cite{lemus2015diseno}, ya que son circuitos electrónicos utilizados en señales de corriente alterna para determinar el momento preciso en que se corta el eje por la señal alterna. Esta referencia es necesaria para poder establecer un punto de partida respecto del cual se puede disparar un tristor con una señal PWM y así poder modular la potencia de este tipo de cargas \cite{revista}.	En un \emph{dimmer}, este detector de cruce por cero permite ajustes del nivel de poder en la corriente eléctrica. Esto es de gran ayuda debido a que la interrupción de la corriente en cualquier otro punto del circuito eléctrico crea un pico de poder potencialmente dañino \cite{pagina_web}.
	\hfill \break %Salto de línea	
	
	%Agregando fuentes de consulta
	\bibliographystyle{unsrtnat} %abbrnat es parecido a APA
	\bibliography{referencias}
    
\end{document}
